\documentclass[a4paper, 11pt]{scrartcl}

\usepackage{ucs}
\usepackage[framemethod=tikz]{mdframed}
\usepackage{float}
\usepackage{subfig}
\usepackage[utf8x]{inputenc}
\usepackage[T1]{fontenc}
\usepackage{graphicx}
\usepackage[automark]{scrlayer-scrpage}
\usepackage[hidelinks]{hyperref}
\usepackage{listings}
\usepackage{booktabs}
\usepackage{tikz}
\usepackage[acronym]{glossaries}
\usepackage{enumitem}
\usepackage{epigraph}
\usepackage[left=2cm, right=2cm, top=2cm, bottom=2cm]{geometry}

\let\oldsection\section
\renewcommand\section{\clearpage\oldsection}

\makeglossaries

\mdfdefinestyle{note-style}{%
rightline=true,
innerleftmargin=10,
innerrightmargin=10,
outerlinewidth=3pt,
topline=false,
rightline=true,
bottomline=false,
skipabove=\topsep,
skipbelow=\topsep
}

\newglossaryentry{nodejs}
{
name=Node.js,
description={Node.js is an open-source, cross-platform, JavaScript library that executes JavaScript code outside of a browser.}
}

\makeindex

\newacronym{arest}{REST}{Representational State Transfer}

\usetikzlibrary{shapes.geometric, arrows}
\tikzstyle{startstop} = [rectangle, rounded corners, minimum width=3cm, minimum height=1cm,text centered, draw=black, fill=red!30]
\tikzstyle{io} = [trapezium, trapezium left angle=70, trapezium right angle=110, minimum width=3cm, minimum height=1cm, text centered, draw=black, fill=blue!30]
\tikzstyle{process} = [rectangle, minimum width=3cm, minimum height=1cm, text centered, draw=black, fill=orange!30]
\tikzstyle{decision} = [diamond, minimum width=3cm, minimum height=1cm, text centered, draw=black, fill=green!30]
\tikzstyle{arrow} = [thick,->,>=stealth]

\pagestyle{scrheadings}
\clearscrheadfoot
\ofoot[\pagemark]{\pagemark}
\ihead[]{\headmark}
\setheadsepline[\textwidth]{0.5pt}

\title{Final Report}
\author{Marc Berli, Simon Stucki, ...}
\date{\today{}, Zürich}

\begin{document}

\begin{titlepage}
\centering
{\scshape\LARGE PSIT4 \par}
\vspace{1cm}
{\scshape ZHAW - School of Engineering\par}
\vspace{1cm}
{\scshape\Large Gl0bis\par}
\vspace{1.5cm}
{\huge\bfseries Schlussbericht\par}
\vspace{2cm}
von
\vspace{1em}
\Large\itshape \\ Raphael Mailänder, Marc Berli, Ferenc Kuntić \\ Michael Schaufelberger, Filip Kašiković, und Simon Stucki\par
\vfill
\textbf{Team}\par
IT17ta\_ZH\par
\vspace{2em}
\textbf{Status}\par
In Progress

\vfill

{\large \today \textbf{ --} Zürich\par}
\end{titlepage}

\tableofcontents

\newpage

\section{Kontext}
% Dieser erste Teil des Dokuments beschreibt grob die Idee des Produktes und das Umfeld des
% Produktes. Dieses Kapitel muss nicht lang sein (halbe bis max. 2 Seiten) und sollte folgende
% Fragen beantworten:
% • Um was geht es bei der Software, dem Produkt, dem System?
% • Was wird produziert?
% • Wie passt es ins bestehende Umfeld?
% • Wer verwendet die Software? (Aktoren, Anwender, Rollen, ...)
% Dieses Kapitel muss in jedem Software-Guidebook enthalten sein.

Die Idee des Projekts ist es, ein Spiel zu entwickeln,
welches sich nicht durch klassische Echtzeiteingaben,
sondern durch Code steuern lässt.
Der Spieler schreibt dazu ein Skript,
welches in einer Schleife ausgeführt wird.
Er kann mittels des Skripts beispielsweise den aktuellen Spielstand auslesen,
Berechnungen anstellen und zum Schluss eine Liste von vordefinierten Aktionen durchführen.

Das Spiel selbst ist in seiner Grundform ein Multiplayer Survival Game.
Es treten mehrere Spieler gegeneinander an.
Ziel des Spiels ist es, der einzige Überlebende zu sein.
Dazu müssen Einheiten der Spieler beispielsweise Essen aufsammeln,
um nicht zu verhungern, Ressourcen sammeln, um Ausrüstung oder Gebäude zu bauen,
oder können gegnerische Einheiten angreifen.
Dieses simple Spielprinzip ist für Spieler einfach zu verstehen oder bereits bekannt.
Die Steuerung durch den Code erlaubt uns,
den Schwerpunkt des Gameplays stärker auf die Vorausplanung und langfristigen Strategien
zu legen und reaktives Gameplay in den Hintergrund zu stellen.

(Eventuell ein 'game play' Bild reinpasten)

Das Spiel wird als Webapplikation umgesetzt.
Spieler können sich gegenseitig herausforden und ein Match starten.
Während des Spielverlaufs können sie ihr Skript aktualisieren und visuell verfolgen,
wie das Spielfeld aktuell aussieht. Zuschauer können den verschiedenen Matches beitreten und zuschauen.

Für die Entwicklung unseres Spieles wird github und azure als Repository Hosting Service verwendet,
während dem die Hauptprogrammiersprache im backend sowie im auch frontend TypeScript zum Einsatz kommt.

Das Spiel richtet sich auf leidenschaftliche und engagierte Software Entwickler aus,
die ihr Können nicht nur beim 'coden', sondern auch beim verständig der KI Architektur zeigen wollen.

\section{Ziele und Hauptfunktonen}
%• Ist allen Teilnehmern klar, was das System, die Software macht?
%• Welche Funktionen sind essentiell für den Erfolg des Produkts
%• Wann ist es ein Erfolg? Welche Ziele müssen erreicht werden? (quantitative technische,
%wirtschaftliche oder organisatorische Ziele)
% Spiel beschreiben und das so kleinwenig das technische ansprechen

Die essenziellen Komponenten des Spieles sind einerseits ein lauffähiger Programmcode, der es erlaubt
die KI auszuführen und andererseits die Grafik, die es dem Benutzer ersichtlich macht, was für Auswirkungen
sein Code hat.
Um diese Ziele zu erreichen, ist es wichtig, eine gut aufgebaute Organisation zu haben, dessen Struktur und
Visionen klar definiert sind. Ein agiles Umfeld wie Scrum wird uns helfen, die stetig auftretenden Probleme
besser und effizienter zu bewältigen.
Das Spiel soll kostenlos als Webapplikation den Benutzern zu Verfügung stehen. Die Einnahmen werden
durch Werbung eingenommen.

\section{Qualitätsattribute}
%• Welche Qualitätsattribute soll die Software/das System erfüllen?
%• Sind die Qualitatsatribute SMART (specific, measurable, achievable, relevant, timely)?
%• Gibt es Attribute, die üblicherweise als gegeben betrachtet werden explizit ausgenommen sind?
%• Sind alle Attribute und Anforderungen realistisch?

Relevanz zu den Spiel Attributen abinden
Qualitätsattribute:
\begin{itemize}

\item Performance - Das Spiel soll den eingegeben Code innerhalb weniger Minuten kompilieren und ausführen
\item Verfügbarkeit - Der Server sollte 24h laufen
\item Security - Eine Prüfung auf schädlichen Code sowie Standard gegen 'Javascript Injection' müssen geprüft werden, zudem muss sich jeder Benutzer authentifizieren, um die Webapplikation zu bedienen
\item Erweiterbarkeit - Die Software muss modular aufgebaut sein, um einzelne Funktionen und Features einfach zu erweitern
\item Zuverlässigkeit - Das Spiel muss gewährleisten, dass es fliessend ohne grössere Unterbrüche läuft und ein fehlerhaftes/schädliches Script nicht ausführ
\item Benutzerfreundlichkeit - Das Userinterface soll es ermöglichen, die gesuchten Infos möglichst schnell, unmittelbar und bequem bereitzustellen und es muss dem User erlauben, dass man sich auf der Seite bewegen kann, ohne auf Probleme zu stossen
\item Failover / Disaster Recovery Ziele - Die Recovery Plans werden eingeschränkt auf:
\begin{enumerate}
\item Hardware Probleme - Es muss sichergestellt werden das bei einem CPU Ausfall
\item Applikations- Code Probleme -
\item Dependency probleme -
\end{enumerate}

\end{itemize}

Um die Qualitätsattribute einzuhalten, ist die Sicherstellung von Programmcode-Entwicklungsprinzipien und Eskalationsplännen ein Muss. Diese werden mithilfe
von agilen Framework Prozessen wie Scrum und zuverlässigen Entwicklung Tools wie Github/Jira oder Confluence einfacher eingehalten.

\section{Bekannte Beschränkungen}
%• Zeit
%• Budget und Ressourcen
%• Technologievorgaben
%• Ziel-Plattform
%• Standardprotoklle
%• Grösse des Entwicklungsteams
%• Kompetenzenprofil des Entwicklungsteams
%• lokale Standards

Bei unserer Software-Entwicklung werden die Prozesse von Story Beschreibung, Entwicklung, Testing und Code-Review befolgt und eingehalten.
Ein strukturiertes Vorgehen ist zwingend, um den Durchbruch bei unserem IT-Projekt zu schaffen.

Unsere Software ist unabhängig von dem Entwicklungsprozess folgenden Beschränkungen ausgesetzt.
\begin{enumerate}
\item Zeit: die Zeitdauer des Projekts ist eingeschränkt was sich auf 6 Sprints beläuft.
\item Budget und Ressourcen: Das Budget soll sich möglichst auf das Minimum beziehen und optimaler Weise jede Ressource zu 100\% ausnutzen.
\item Technologien:
	\begin{enumerate}

	\item Cloud basierende Hosting Services: Github und Azure
	\item TypeScript:

	\end{enumerate}
\item Grösse des Entwicklungsteams: Das Team besteht aus einem Product Owner, einem Scrum master und 4 weiteren Entwicklern
\item Kompetenzenprofil: Alle Entwickler haben bereits mehrjährige Erfahrungen mit den Technologien, welche für unsere Entwicklung eingesetzt werden.
\end{enumerate}

\section{Verwendete Prinzipien}

\section{Architektur}

Der Aufbau unserer Software hat in voraus bereits eine Grobplanung in welchen schritten und Phasen entwickelt wird.
Dies ermöglicht es uns, bereits frühzeitig Fehler zu erkennen und Schnittstellen genauer zu spezifizieren.

\section{Externe Schnittstellen}

\section{Code}

\section{Data}

\section{Infrastructure Architecture}

\section{Installation}

\section{Operation und Support}

\section{Entscheidungs-Logbuch}

\begin{itemize}
\item 21.02.2020
\begin{itemize}
\item Wir haben auf das Genre "Survival" geeinigt. Dies, da es ein sehr flexibles Genre ist, mit dem bereits durch eine sehr einfache Spielmechanik (Hunger) das Grundprinzip umgesetzt werden kann.
\item Wir haben uns entschieden, die Steuerung mittels Code zu machen, da es erst wenige Spiele mit einer solchen Steuerung gibt und uns die technischen sowie auch Gameplay-spezifischen Aspekte interessieren.
\item Wir setzen Frontend und Backend mittels JavaScript (TypeScript) um, da unser Know-how mit dieser Sprache am höchsten ist und sie uns hohe Flexibilität bietet.
\end{itemize}
\end{itemize}

\clearpage

\printglossary[type=\acronymtype]

\printglossary

\end{document}
