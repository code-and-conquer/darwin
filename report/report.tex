\documentclass[a4paper, 11pt]{scrartcl}

\usepackage{ucs}
\usepackage[framemethod=tikz]{mdframed}
\usepackage{float}
\usepackage{subfig}
\usepackage[utf8x]{inputenc}
\usepackage[T1]{fontenc}
\usepackage{graphicx}
\usepackage[automark]{scrlayer-scrpage}
\usepackage[hidelinks]{hyperref}
\usepackage{listings}
\usepackage{booktabs}
\usepackage{tikz}
\usepackage[acronym]{glossaries}
\usepackage{enumitem}
\usepackage{epigraph}
\usepackage[left=2cm, right=2cm, top=2cm, bottom=2cm]{geometry}

\let\oldsection\section
\renewcommand\section{\clearpage\oldsection}

\makeglossaries

\mdfdefinestyle{note-style}{%
  rightline=true,
  innerleftmargin=10,
  innerrightmargin=10,
  outerlinewidth=3pt,
  topline=false,
  rightline=true,
  bottomline=false,
  skipabove=\topsep,
  skipbelow=\topsep
}

\newglossaryentry{nodejs}
{
    name=Node.js,
    description={Node.js is an open-source, cross-platform, JavaScript library that executes JavaScript code outside of a browser.}
}

\makeindex

\newacronym{arest}{REST}{Representational State Transfer}

\usetikzlibrary{shapes.geometric, arrows}
\tikzstyle{startstop} = [rectangle, rounded corners, minimum width=3cm, minimum height=1cm,text centered, draw=black, fill=red!30]
\tikzstyle{io} = [trapezium, trapezium left angle=70, trapezium right angle=110, minimum width=3cm, minimum height=1cm, text centered, draw=black, fill=blue!30]
\tikzstyle{process} = [rectangle, minimum width=3cm, minimum height=1cm, text centered, draw=black, fill=orange!30]
\tikzstyle{decision} = [diamond, minimum width=3cm, minimum height=1cm, text centered, draw=black, fill=green!30]
\tikzstyle{arrow} = [thick,->,>=stealth]

\pagestyle{scrheadings}
\clearscrheadfoot
\ofoot[\pagemark]{\pagemark}
\ihead[]{\headmark}
\setheadsepline[\textwidth]{0.5pt}

\title{Final Report}
\author{Marc Berli, Simon Stucki, ...}
\date{\today{}, Zürich}

\begin{document}

\begin{titlepage}
	\centering
	{\scshape\LARGE PSIT4 \par}
  \vspace{1cm}
  {\scshape ZHAW - School of Engineering\par}
	\vspace{1cm}
	{\scshape\Large Gl0bis\par}
	\vspace{1.5cm}
	{\huge\bfseries Schlussbericht\par}
	\vspace{2cm}
  von
	\vspace{1em}
  \Large\itshape \\ Raphael Mailänder, Marc Berli, Ferenc Kuntic \\ Michael Schaufelberger, Filip Kasikovic, und Simon Stucki\par
	\vfill
  \textbf{Team}\par
  IT17ta\_ZH\par
	\vspace{2em}
  \textbf{Status}\par
	In Progress

	\vfill

	{\large \today \textbf{ --} Zürich\par}
\end{titlepage}

\tableofcontents

\newpage

\section{Kontext}
% Dieser erste Teil des Dokuments beschreibt grob die Idee des Produktes und das Umfeld des
% Produktes. Dieses Kapitel muss nicht lang sein (halbe bis max. 2 Seiten) und sollte folgende
% Fragen beantworten:
% • Um was geht es bei der Software, dem Produkt, dem System?
% • Was wird produziert?
% • Wie passt es ins bestehende Umfeld?
% • Wer verwendet die Software? (Aktoren, Anwender, Rollen, ...)
% Dieses Kapitel muss in jedem Software-Guidebook enthalten sein.

Die Idee des Projekts ist es ein Spiel zu entwickeln, welches sich nicht durch klassische Echtzeiteingaben, sondern durch Code steuern lässt. Der Spieler schreibt dazu ein Skript, welches in einer Schleife ausgeführt wird. Er kann mittels des Skripts beispielsweise den aktuellen Spielzustand auslesen, Berechnungen anstellen und zum Schluss eine Liste von vordefinierten Aktionen durchführen.

Das Spiel selbst ist in seiner Grundform ein Multiplayer Survival Game. Es treten mehrere Spieler gegeneinander an. Ziel des Spiels ist es, der einzige Überlebende zu sein. Dazu müssen  Einheiten der Spieler beispielsweise Essen aufsammeln, um nicht zu verhungern, Ressourcen sammeln, um Ausrüstung oder Gebäude zu bauen, oder können gegnerische Einheiten angreifen. Dieses simple Spielprinzip ist für Spieler einfach zu verstehen oder bereits bekannt. Die Steuerung durch den Code erlaubt uns, den Schwerpunkt des Gameplays stärker auf die Vorausplanung und langfristigen Strategien zu legen und reaktives Gameplay in den Hintergrund zu stellen.

Das Spiel wird als Webapplikation umgesetzt. Spieler können sich gegenseitig herausforden und ein Match starten. Während des Spielverlaufs können sie ihr Skript aktualisieren und visuell verfolgen, wie das Spielfeld aktuell aussieht. Zuschauer können die verschiedenen Matches beitreten und zuschauen.

\section{Ziele und Hauptfunktonen}

\section{Bekannte Beschränkungen}

\section{Verwendete Prinzipien}

\section{Architektur}

\section{Externe Schnittstellen}

\section{Code}
\subsection{DoD}
\begin{itemize}
\item Alle Akzeptanzkriterien werden erfüllt
\item Der Code ist fertiggestellt und im Versionierungssystem eingespielt
\item Dokumentation aktualisiert
\item Es wurde ein Code Review durchgeführt oder der Code wurde im Pair Programming erarbeitet
\item Coding Guidelines und Standards wurden eingehalten
\item Unit Tests wurden erfolgreich durchgeführt 
\item Es sind keine kritischen Bugs offen
\item "Functional Tests" erfolgreich durchlaufen
\end{itemize}
\section{Data}

\section{Infrastructure Architecture}

\section{Installation}

\section{Operation und Support}

\section{Entscheidungs-Logbuch}

\begin{itemize}
  \item 21.02.2020
  \begin{itemize}
    \item Wir haben uns zum Genre "Survival" geeinigt. Dies, da es ein sehr flexibles Genre ist, mit dem bereits durch eine sehr einfache Spielmechanik (Hunger) das Grundprinzip umgesetzt werden kann.
    \item Wir haben uns entschieden die Steuerung mittels Code zu machen, da es erst wenige Spiele mit einer solchen Steuerung gibt und uns die technischen sowie auch gameplay-spezifischen Aspekte interessieren.
    \item Wir setzen Frontend und Backend mittels JavaScript (TypeScript) um, da unser Know-How mit dieser Sprache am höchsten ist und sie uns hohe Flexibilität bietet.
  \end{itemize}
\end{itemize}

\clearpage

\printglossary[type=\acronymtype]

\printglossary

\end{document}
